estrutura-yves.tex
\documentclass[
	% -- opções da classe memoir --
	12pt,				% tamanho da fonte
	openright,			% capítulos começam em pág ímpar (insere página vazia caso preciso)
	twoside,			% para impressão em recto e verso. Oposto a oneside
	a4paper,			% tamanho do papel. 
	% -- opções da classe abntex2 --
	%chapter=TITLE,		% títulos de capítulos convertidos em letras maiúsculas
	%section=TITLE,		% títulos de seções convertidos em letras maiúsculas
	%subsection=TITLE,	% títulos de subseções convertidos em letras maiúsculas
	%subsubsection=TITLE,% títulos de subsubseções convertidos em letras maiúsculas
	% -- opções do pacote babel --
	english,			% idioma adicional para hifenização
	french,				% idioma adicional para hifenização
	spanish,			% idioma adicional para hifenização
	brazil				% o último idioma é o principal do documento
  ]{abntex2}

\usepackage{hyperref}

\begin{document}
  
\chapter{Introdução}\label{cap_01_introducao}

\section{Objetivos}

\subsection{Objetivos gerais}

(eu vou relatar o que exatemente?)

Relatar o projeto dos microservicos do app deepgammon

\subsection{Objetivos específicos}


\section{Metodologia}

\section{Delimitação do trabalho}

\section{Organização do trabalho}



\chapter{Docker e microserviços}\label{cap_02_microservico}



\chapter{Os microserviços do projeto Gamão}\label{cap_02_gamao}


\end{document}
